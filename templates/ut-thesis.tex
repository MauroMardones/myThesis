%% ut-thesis.tex -- document template for graduate theses at UofT
%%
%% Copyright (c) 1998-2013 Francois Pitt <fpitt@cs.utoronto.ca>
%% last updated at 16:20 (EDT) on Wed 25 Sep 2013
%%
%% This work may be distributed and/or modified under the conditions of
%% the LaTeX Project Public License, either version 1.3c of this license
%% or (at your option) any later version.
%% The latest version of this license is in
%%     http://www.latex-project.org/lppl.txt
%% and version 1.3c or later is part of all distributions of LaTeX
%% version 2005/12/01 or later.
%%
%% This work has the LPPL maintenance status "maintained".
%%
%% The Current Maintainer of this work is
%% Francois Pitt <fpitt@cs.utoronto.ca>.
%%
%% This work consists of the files listed in the accompanying README.

%% SUMMARY OF FEATURES:
%%
%% All environments, commands, and options provided by the `ut-thesis'
%% class will be described below, at the point where they should appear
%% in the document.  See the file `ut-thesis.cls' for more details.
%%
%% To explicitly set the pagestyle of any blank page inserted with
%% \cleardoublepage, use one of \clearemptydoublepage,
%% \clearplaindoublepage, \clearthesisdoublepage, or
%% \clearstandarddoublepage (to use the style currently in effect).
%%
%% For single-spaced quotes or quotations, use the `longquote' and
%% `longquotation' environments.


%%%%%%%%%%%%         PREAMBLE         %%%%%%%%%%%%

%%  - Default settings format a final copy (single-sided, normal
%%    margins, one-and-a-half-spaced with single-spaced notes).
%%  - For a rough copy (double-sided, normal margins, double-spaced,
%%    with the word "DRAFT" printed at each corner of every page), use
%%    the `draft' option.
%%  - The default global line spacing can be changed with one of the
%%    options `singlespaced', `onehalfspaced', or `doublespaced'.
%%  - Footnotes and marginal notes are all single-spaced by default, but
%%    can be made to have the same spacing as the rest of the document
%%    by using the option `standardspacednotes'.
%%  - The size of the margins can be changed with one of the options:
%%     . `narrowmargins' (1 1/4" left, 3/4" others),
%%     . `normalmargins' (1 1/4" left, 1" others),
%%     . `widemargins' (1 1/4" all),
%%     . `extrawidemargins' (1 1/2" all).
%%  - The pagestyle of "cleared" pages (empty pages inserted in
%%    two-sided documents to put the next page on the right-hand side)
%%    can be set with one of the options `cleardoublepagestyleempty',
%%    `cleardoublepagestyleplain', or `cleardoublepagestylestandard'.
%%  - Any other standard option for the `report' document class can be
%%    used to override the default or draft settings (such as `10pt',
%%    `11pt', `12pt'), and standard LaTeX packages can be used to
%%    further customize the layout and/or formatting of the document.

%% Updated by MTW
%% This code allows users to specify named documentclass options, including
%% draft, fontsize, margins, spacing, clearpagestyle, notespacing
%% They may also specify any other class options using the 'classoption'
%% statement which is recycled by the for loop
\documentclass[
$if(draft)$
  draft,
$endif$
$if(fontsize)$
  $fontsize$,
$endif$
$if(margins)$
  $margins$,
$endif$
$if(spacing)$
  $spacing$,
$endif$
$if(clearpagestyle)$
  $clearpagestyle$,
$endif$
$if(notespacing)$
  $notespacing$,
$endif$
$for(classoption)$
  $classoption$$sep$,
$endfor$
]{templates/ut-thesis}

%% Additional packages and code to work with R Markdown
%% Not sure if this should go in the class file instead
%% Additional packages offer native support for common functionality
\usepackage{appendix}
\usepackage{graphicx,latexsym}
\usepackage{amsmath}
\usepackage{amssymb,amsthm}
\usepackage{longtable,booktabs,setspace}
\usepackage[hyphens]{url}
\usepackage[colorlinks = true,
			urlcolor = blue,
			linkcolor = blue,
			citecolor = blue,
			anchorcolor = black]{hyperref}
\usepackage{lmodern}
\usepackage{float}
\usepackage{array}
\floatplacement{figure}{H}
\usepackage{tocloft}
\usepackage{parskip}
\usepackage{microtype}
\usepackage{nameref}
\usepackage{lscape}

%% Required by pandoc now
%% See https://github.com/jgm/pandoc/blob/master/data/templates/default.latex#L340-L364
$if(csl-refs)$
\newlength{\cslhangindent}
\setlength{\cslhangindent}{1.5em}
\newlength{\csllabelwidth}
\setlength{\csllabelwidth}{3em}
\newlength{\cslentryspacingunit} % times entry-spacing
\setlength{\cslentryspacingunit}{\parskip}
\newenvironment{CSLReferences}[2] % #1 hanging-ident, #2 entry spacing
 {% don't indent paragraphs
  \setlength{\parindent}{0pt}
  % turn on hanging indent if param 1 is 1
  \ifodd #1
  \let\oldpar\par
  \def\par{\hangindent=\cslhangindent\oldpar}
  \fi
  % set entry spacing
  \setlength{\parskip}{#2\cslentryspacingunit}
 }%
 {}
\usepackage{calc}
\newcommand{\CSLBlock}[1]{#1\hfill\break}
\newcommand{\CSLLeftMargin}[1]{\parbox[t]{\csllabelwidth}{#1}}
\newcommand{\CSLRightInline}[1]{\parbox[t]{\linewidth - \csllabelwidth}{#1}\break}
\newcommand{\CSLIndent}[1]{\hspace{\cslhangindent}#1}
$endif$

\usepackage{rotating} % Package added to allow the rotation of figures and chart on a page, {sidewaysfigure} command
\usepackage{tablefootnote} % Packaged added to allow footnotes in the tabular
                           % environment, use \tablefootnote command

%% Added to fix issue with Pandoc not rendering lists properly
\providecommand{\tightlist}{%
  \setlength{\itemsep}{0pt}\setlength{\parskip}{0pt}}
%% end add

%% Added by MTW to provide proper Appendix functionality
%% This package produces a warning for being 'alpha'; a bit worrisome
%% This code block creates List of Appendices, and ensures only top-level
%% appendix titles are printed to ToC; this can be changed based on SGS guide
\usepackage{silence}
\WarningsOff[scrwfile] % silence package startup warnings
\usepackage{scrwfile}

\TOCclone[List of Appendices]{toc}{atoc}
\addtocontents{atoc}{\protect\value{tocdepth}=-1}
\newcommand\listofappendices{\listofatoc}

\newcommand*\savedtocdepth{}
\AtBeginDocument{%
  \edef\savedtocdepth{\the\value{tocdepth}}%
}

\let\originalappendix\appendix
\renewcommand\appendix{%
  \originalappendix
  \cleardoublepage
  \addcontentsline{toc}{chapter}{Appendices}%
  \addtocontents{toc}{\protect\value{tocdepth}=-1}%
  \addtocontents{atoc}{\protect\value{tocdepth}=\savedtocdepth}%
}


%% Control LOF appearance
\newlength{\mylen}
\renewcommand{\cftfigpresnum}{\figurename\enspace}
\settowidth{\mylen}{\cftfigpresnum\cftfigaftersnum}
\addtolength{\cftfignumwidth}{\mylen}

%% Control LOT appearance
\renewcommand{\cfttabpresnum}{\tablename\enspace}
\settowidth{\mylen}{\cfttabpresnum\cfttabaftersnum}
\addtolength{\cfttabnumwidth}{\mylen}

%% Custom command to get current chapter name
\makeatletter
\newcommand*{\currentname}{ \@currentlabelname}

%% Change ToC title
\renewcommand{\contentsname}{Table of Contents}

%% The standard packages `geometry' and `setspace' are already loaded by
%% `ut-thesis' -- see their documentation for details of the features
%% they provide.  In particular, you may use the \geometry command here
%% to adjust the margins if none of the ut-thesis options are suitable
%% (see the `geometry' package for details).  You may also use the
%% \setstretch command to set the line spacing to a value other than
%% single, one-and-a-half, or double spaced (see the `setspace' package
%% for details).


%%%%%%%%%%%%%%%%%%%%%%%%%%%%%%%%%%%%%%%%%%%%%%%%%%%%%%%%%%%%%%%%%%%%%%%%
%%                                                                    %%
%%                   ***   I M P O R T A N T   ***                    %%
%%                                                                    %%
%%  Fill in the following fields with the required information:       %%
%%   - \degree{...}       name of the degree obtained                 %%
%%   - \department{...}   name of the graduate department             %%
%%   - \gradyear{...}     year of graduation                          %%
%%   - \author{...}       name of the author                          %%
%%   - \title{...}        title of the thesis                         %%
%%%%%%%%%%%%%%%%%%%%%%%%%%%%%%%%%%%%%%%%%%%%%%%%%%%%%%%%%%%%%%%%%%%%%%%%

%% *** Change this example to appropriate values. ***

%% Updated by MTW to change content to variables
%% These variables are specified by the YAML header in RMarkdown file
\degree{$degree$}
\department{$department$}
\gradyear{$gradyear$}
\author{$author$}
\title{$title$}

%% *** NOTE ***
%% Put here all other formatting commands that belong in the preamble.

%% Add by MTW
%% This code allows users to reference additonal files in RMarkdown header
%% containing LaTeX code; contents of said file is injected here
$for(header-includes)$
  $header-includes$
$endfor$


%% *** Adjust the following settings as desired. ***

%% List only down to subsections in the table of contents;
%% 0=chapter, 1=section, 2=subsection, 3=subsubsection, etc.

%% Updated by MTW; specified by function call in R; adjustable
\setcounter{tocdepth}{$toc-depth$}

%% Make each page fill up the entire page.
\flushbottom


%%%%%%%%%%%%      MAIN  DOCUMENT      %%%%%%%%%%%%

\begin{document}

%% This sets the page style and numbering for preliminary sections.
\begin{preliminary}

%% This generates the title page from the information given above.
\maketitle

%% There should be NOTHING between the title page and abstract.
%% However, if your document is two-sided and you want the abstract
%% _not_ to appear on the back of the title page, then uncomment the
%% following line.
\clearpage
\thispagestyle{empty}
\mbox{}
\newpage


%% This generates an "acknowledgements" section, if needed
%% (uncomment to have it appear in the document).
\begin{acknowledgements}
\addcontentsline{toc}{chapter}{Acknowledgements} % Added by MTW
$acknowledgements$ % Added by MTW
%% *** Put your Acknowledgements here. ***
\end{acknowledgements}


%% This generates an "dedication" section, if needed
%% (uncomment to have it appear in the document).
\begin{dedication}
\addcontentsline{toc}{chapter}{Dedication} % Added by MTW
$dedication$ % Added by MTW
% \vspace*{22\baselineskip}
% \begin{flushright}
% \textit{"El lujo es vulgaridad dijo, y me conquistó..."}\\
% \textit{(Patricio Rey y sus Redonditos de Ricota)}\\
% \vspace*{0.5cm}
% \textit{To María José, always...} \\
% \textit{for her boundless kindness and simplicity, which illuminate my days, time and time again.}
% \end{flushright}
\end{dedication}


%% This generates the abstract page, with the line spacing adjusted
%% according to SGS guidelines.
\begin{declaration}
\addcontentsline{toc}{chapter}{Declaration}
$declaration$ % Added by MTW
This is to declarate that the intellectual content of this thesis is the product of my own work and that all the assistance received in preparing this thesis and sources have been acknowledged.
\end{declaration}


% %% This generates the abstract page, with the line spacing adjusted
% %% according to SGS guidelines.
% \begin{abstract}
% \addcontentsline{toc}{chapter}{Abstract}
% $abstract$ % Added by MTW
% %% *** Put your Abstract here. ***
% %% (At most 150 words for M.Sc. or 350 words for Ph.D.)
% \end{abstract}

%% Anything placed between the abstract and table of contents will
%% appear on a separate page since the abstract ends with \newpage and
%% the table of contents starts with \clearpage.  Use \cleardoublepage
%% for anything that you want to appear on a right-hand page.

%% This generates a "dedication" section, if needed -- just a paragraph
%% formatted flush right (uncomment to have it appear in the document).
%\begin{dedication}
%% *** Put your Dedication here. ***
%\end{dedication}

%% The `dedication' and `acknowledgements' sections do not create new
%% pages so if you want the two sections to appear on separate pages,
%% uncomment the following line.
%\newpage  % separate pages for dedication and acknowledgements

%% Alternatively, if you leave both on the same page, it is probably a
%% good idea to add a bit of extra vertical space in between the two --
%% for example, as follows (adjust as desired).
%\vspace{.5in}  % vertical space between dedication and acknowledgements

\clearpage
\thispagestyle{empty}
\mbox{}
\newpage


%% Table of Contents
\tableofcontents
\newpage

%% Added by MTW
%% Allows users to toggle whether Lists of Tables/Figures are included
%% in the TOC; also whether to include List of Appendices

%% List of Figures
\phantomsection\addcontentsline{toc}{chapter}{\listfigurename}
\listoffigures
\newpage

%% List of Tables
\phantomsection\addcontentsline{toc}{chapter}{\listtablename}
\listoftables
\newpage

%% List of Appendices
$if(list-of-app)$
  \phantomsection\addcontentsline{toc}{chapter}{List of Appendices}
  \listofappendices
  \newpage
$endif$

%% List of Abbreviations.
$if(list-of-abbr)$
  $list-of-abbr$
$endif$

%% End of the preliminary sections: reset page style and numbering.
\end{preliminary}

%% *** Chapters go here. ***

% Content from before_body will go here
$for(include-before)$
  $include-before$
$endfor$

% Body content of the RMarkdown injected here
$body$

% Content from after_body will go here
$for(include-after)$
  $include-after$
$endfor$

\end{document}
